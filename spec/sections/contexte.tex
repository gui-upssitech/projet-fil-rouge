\documentclass[../main.tex]{subfiles}
\graphicspath{{assets/}{../assets/}}

\begin{document}

    \section{Contexte}
    \subsection{Description entreprise}
    \paragraph{}
    \textbf{OmniWare} est une \textbf{entreprise française} spécialisée dans la conception et le développement de \textbf{logiciels informatiques} pour les secteurs de l’\textbf{aéronautique}, du \textbf{spatial}, de l’\textbf{automobile} et, plus généralement, de la \textbf{robotique}. En 2019, elle réalisait un chiffre d’affaires d’environ \textbf{45 millions d’euros} et employait plus \textbf{350 employés} dans plusieurs établissements dont le principal se situe à \textbf{Toulouse}. La société est devenue en 2020 une filiale de la société japonaise \textbf{ZMP} dont le siège social est à Tokyo.
    \subsection{Objectif}
    \paragraph{}
    De nos jours, l’accès à l’information est un enjeu essentiel. Il faut donc pouvoir accéder à une grande quantité de documents mis à disposition (Internet, archives…) à l’aide d’une description synthétique de leurs contenus. Pour cela, nous devrons passer d’abord par une phase d’indexation automatique.

    \paragraph{}
    Les documents traités seront de nature diverse et seront enregistrés dans un format standard prédéfini. La description synthétique d’un document se basera sur sa nature et son contenu. Cette description sera ensuite utilisée par un moteur de recherche.

    \paragraph{}
    L’utilisateur ne connaît pas le fonctionnement du moteur de recherche, ni la méthode d’indexation des données, le logiciel doit être simple d’utilisation. Toute la phase de configurations, paramétrages et méthodes est gérée par les administrateurs.

\end{document}