\documentclass[../main.tex]{subfiles}
\graphicspath{{\subfix{../assets/}}}

\begin{document}

\section{Contexte}
\subsection{Description entreprise}
\paragraph{}
\textbf{OmniWare} est une \textbf{entreprise française} spécialisée dans la conception et le développement de \textbf{logiciels informatiques} pour les secteurs de l’\textbf{aéronautique}, du \textbf{spatial}, de l’\textbf{automobile} et, plus généralement, de la \textbf{robotique}. En 2019, elle réalisait un chiffre d’affaires d’environ \textbf{45 millions d’euros} et employait plus \textbf{350 employés} dans plusieurs établissements dont le principal se situe à \textbf{Toulouse}. La société est devenue en 2020 une filiale de la société japonaise \textbf{ZMP} dont le siège social est à Tokyo.
\subsection{Objectif}
\paragraph{}
De nos jours, l’accès à l’information est un enjeu essentiel. Il faut donc pouvoir accéder à une grande quantité de documents mis à disposition (Internet, archives…) à l’aide d’une description synthétique de leurs contenus. Pour cela, nous devrons passer d’abord par une phase d’indexation automatique.

\paragraph{}
Les documents traités seront de nature diverse et seront enregistrés dans un format standard prédéfini. La description synthétique d’un document se basera sur sa nature et son contenu. Cette description sera ensuite utilisée par un moteur de recherche.

\paragraph{}
L’utilisateur ne connaît pas le fonctionnement du moteur de recherche, ni la méthode d’indexation des données, le logiciel doit être simple d’utilisation. Toute la phase de configurations, paramétrages et méthodes est gérée par les administrateurs.
\newpage

\section{Description générale du logiciel}
\subsection{Environnement}

\paragraph{}
L’environnement dans lequel sera développé le moteur de recherche est le système d’exploitation Linux. Plus particulièrement, le développement se fera sous le langage C dans un premier temps. L’environnement doit alors supporter le langage C afin de faire fonctionner l’application sur la machine.

\paragraph{}
Nous retrouverons aussi l’utilisateur qui va pouvoir interagir avec les entrées (clavier…) et sorties du système (écran…) afin de communiquer avec le logiciel.

\subsection{Composantes du logiciel et acteurs}

\paragraph{}
Une base de document est mise à disposition et contient des documents de type Texte, de type Image et de type Audio. Ainsi, il y aura une base de descripteurs pour chaque type de document traité, qui sauvegardera chaque descripteur.

\paragraph{}
Au sein de l’application, nous avons deux acteurs :
\begin{itemize}
    \item L’\underline{utilisateur} doit pouvoir formuler une requête via le moteur de recherche et ainsi consulter le résultat de celle-ci.
    \item L’\underline{administrateur}, en plus d’hériter de toutes les fonctionnalités de l’utilisateur, doit pouvoir configurer le processus d’indexation, lancer l’indexation et visualiser les descripteurs.
\end{itemize}

\subsection{Fonctionnalités}
\paragraph{}
Le logiciel se décompose donc en deux grandes parties fonctionnelles :
\begin{itemize}
    \item Une partie d’indexation.
    \item Une partie de comparaison.
\end{itemize}

\paragraph{}
La dernière partie sera l'intégration des éléments précédents sous la forme d’un moteur de recherche qui comparera et triera les résultats en fonction de paramètres et de configurations pour ensuite afficher à l’utilisateur les documents les plus proches de la recherche.

\paragraph{}
Plusieurs types de documents sont pris en compte dans ce projet dans des extensions spécifiques (voir table ci-dessous).  	

\begin{figure}[h]
    \begin{center}
        \begin{tabular}{|c|c|c|}
            \hline
            \rowcolor{Gray}Type de document & Extension & Version .txt disponible \\
            \hline
            Texte & .xml & Non \\
            \hline
            Image couleur & .jpg & Oui (matrices intensités des pixels = RGB) \\
            \hline
            Image noir et blanc & .bmp & Oui (matrice intensités des pixels niveau de gris) \\
            \hline
            Audio & .wav .bin &  Oui (valeurs des échantillons du signal numérisé) \\
            \hline
        \end{tabular}
    \end{center}
    \caption{Tableau récapitulatif des différents types de données pris en charge}
\end{figure}

% TO DO
\paragraph{}
Voici un diagramme des cas d’utilisation qui reprend toutes les fonctionnalités évoquées précédemment : 

\end{document}