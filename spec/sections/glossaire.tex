\documentclass[../main.tex]{subfiles}
\graphicspath{{assets/}{../assets/}}

\begin{document}
    
    \section{Glossaire}
    \paragraph{Balise}
    Une balise est un nom commode pour désigner les constructions entre deux chevrons (<, >) dans un fichier XML. Nous distinguons globalement les balises ouvrantes <élément attribut="valeur"> et les balises fermantes </élément> (sans attributs et commençant par une barre oblique).

    \paragraph{Token}
    Un token est une chaîne de caractères finissant par '\textbackslash 0' en langage C. Un texte est une suite de tokens séparés par des espaces. Nous considérons qu’un token a une taille maximale. Seuls les tokens correspondant à des mots pertinents seront retenus.

    \paragraph{Descripteur}
    Représentation synthétique d’un document par diverses informations. Elle varie selon le type du document et les paramètres choisis par l’administrateur.

    \paragraph{Interface}
    Dispositif qui permet la communication entre un utilisateur et le logiciel.

\end{document}

