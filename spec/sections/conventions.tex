\documentclass[../main.tex]{subfiles}
\graphicspath{{assets/}{../assets/}}

\begin{document}

    \section{Conventions de codage}

    \paragraph{}
    Des conventions/normes de codage vont être instaurées pour le projet afin d'uniformiser et rendre le code plus propres et compréhensif. Concernant la typologie du code, nous allons utilisé le “\code{camelCase}” dans laquelle chaque mot d'un mot composé est mis en majuscule, à l'exception du premier mot.

    \paragraph{Exemples}
    \begin{itemize}
        \item Variables: \code{isIndexable} - \code{count}
        \item Nom de fonction: \code{isValid()} - \code{getInformationText()} - \code{pileToFile()} 
    \end{itemize}

    \paragraph{}
    Pour le langage utilisé dans le code,pour les noms de variables/fonctions et pour les commentaires notamment, nous avons choisi de tout mettre en anglais.

    \paragraph{}
    Pour maintenir un code propre parmis les différents développeurs, nous allons utiliser un code formateur “Prettier” (\url{https://prettier.io/}), sur Visual Studio Code, c’est simplement une extension à ajouter sur l'éditeur. Avec un simple raccourci clavier configurable, il vérifie et corrige les indentations selon la configuration qu’on lui aura transmise.

\end{document}